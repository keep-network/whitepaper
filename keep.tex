%% Based on the style files for ACL-2015

\documentclass[11pt]{article}
\usepackage{acl2015}

\title{The Keep Network: A Privacy Layer for Public Blockchains}

\author{Matt Luongo \\
  Fold \\
  {\tt mhluongo@gmail.com}}

\date{}

\begin{document}
\maketitle
\begin{abstract}
  We introduce the keep, a new privacy primitive for smart contracts
  on public blockchains, as well as supporting components, including
  the keep market and token, that make up the Keep network.

  Rather than aiming to solve generic private smart contracts with an
  alternative computation network, we present an incremental approach
  that can be brought to market on the Ethereum public network,
  iterated on, and adapted for other public blockchains and
  cross-blockchain use.
\end{abstract}

\section{Motivation}

\subsection{The irony of public blockchains}

Public blockchains have brought unprecedented transparency and
auditability to financial technology. Records are immutable,
verifiable, and censorship-resistant.

Unfortunately, these strengths are also weaknesses for many potential
users.

For every financial use case a public blockchain enables, its public
status restricts another. Bitcoin is touted as a more private payment
method than the traditional financial system, but those familiar with
the technology know that while it may be censorship-resistant, it’s
certainly not private by default [bitcoin-privacy]. Developers
introduced to Ethereum quickly learn to adjust their expectations
[ethereum-stackexchange]- all contract state is published to the
blockchain, and can be easily read by competing interests.

These issues are recognized by developers of the Bitcoin and Ethereum
projects. Confidential transactions [confidential-transactions] is an
ongoing effort to bring better privacy, and therefore fungibility, to
Bitcoin. As early as December 2014, Vitalik Buterin, one of the
founders of Ethereum, explored solving this problem with secure multi
party computation (sMPC) [secret-sharing-daos]. In more recent
writing, Buterin shares that “...when [he] and others talk to
companies about building their applications on a blockchain, two
primary issues always come up: scalability and privacy”
[privacy-on-the-blockchain].

Scalability of public blockchains is a hurdle to mainstream adoption.
Some of the best minds in the cryptocurrency space [lightning]
[ethereum-sharding] [plasma] are working on multiple order-of-magnitude
improvements. Privacy, however, hasn’t garnered the same attention,
especially in smart contracts.

Basic use cases of smart contracts, including publishing secrets after
certain criteria are met, assessing borrower risk for a loan, and
verifying real-world identity on the blockchain, are incredibly
difficult on today’s public blockchains.

\section{Introducing keeps}

To solve this mismatch between the transparency of public blockchains,
and the need of many smart contracts for private data, we introduce
the idea of {\em keeps}.

A keep is an off-chain container for private data. Keeps allow
contracts to manage and use private data without exposing the data to
the public blockchain.

\subsection{Keep operations}

\subsubsection{Creation and population}
\subsubsection{Publishing data on-chain}
\subsubsection{Access management}
\subsubsection{Destruction}

\section{Managing third-party risk}

\subsection{Secure multi party computation}

\section{Keep providers}
\subsection{Simple sMPC}
\subsection{Signing sMPC}
\subsection{Trusted third-party}
\subsubsection{Redundant trusted third-party}
\subsection{Future providers}

\section{Incentivizing keep providers}

\subsection{Paying for keeps}
\subsection{Concerns with uptimes and reliability}
\subsection{Concerns with active attacks}

\section{High-level network design}

\section{The keep market}
\section{The Keep network token}
\section{The result registry}

\section{Applications}

\section{Dead man switch}
\section{Marketplaces for digital goods}
\section{Pseudorandomness Oracle}
\section{Decentralized signing service}
\section{Encryption service for blockchain storage}
\section{Banking on public blockchains}

\end{document}
